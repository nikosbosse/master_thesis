% Options for packages loaded elsewhere
\PassOptionsToPackage{unicode}{hyperref}
\PassOptionsToPackage{hyphens}{url}
%
\documentclass[
]{book}
\usepackage{lmodern}
\usepackage{amssymb,amsmath}
\usepackage{ifxetex,ifluatex}
\ifnum 0\ifxetex 1\fi\ifluatex 1\fi=0 % if pdftex
  \usepackage[T1]{fontenc}
  \usepackage[utf8]{inputenc}
  \usepackage{textcomp} % provide euro and other symbols
\else % if luatex or xetex
  \usepackage{unicode-math}
  \defaultfontfeatures{Scale=MatchLowercase}
  \defaultfontfeatures[\rmfamily]{Ligatures=TeX,Scale=1}
\fi
% Use upquote if available, for straight quotes in verbatim environments
\IfFileExists{upquote.sty}{\usepackage{upquote}}{}
\IfFileExists{microtype.sty}{% use microtype if available
  \usepackage[]{microtype}
  \UseMicrotypeSet[protrusion]{basicmath} % disable protrusion for tt fonts
}{}
\makeatletter
\@ifundefined{KOMAClassName}{% if non-KOMA class
  \IfFileExists{parskip.sty}{%
    \usepackage{parskip}
  }{% else
    \setlength{\parindent}{0pt}
    \setlength{\parskip}{6pt plus 2pt minus 1pt}}
}{% if KOMA class
  \KOMAoptions{parskip=half}}
\makeatother
\usepackage{xcolor}
\IfFileExists{xurl.sty}{\usepackage{xurl}}{} % add URL line breaks if available
\IfFileExists{bookmark.sty}{\usepackage{bookmark}}{\usepackage{hyperref}}
\hypersetup{
  pdftitle={A Minimal Book Example},
  pdfauthor={Nikos Bosse},
  hidelinks,
  pdfcreator={LaTeX via pandoc}}
\urlstyle{same} % disable monospaced font for URLs
\usepackage{longtable,booktabs}
% Correct order of tables after \paragraph or \subparagraph
\usepackage{etoolbox}
\makeatletter
\patchcmd\longtable{\par}{\if@noskipsec\mbox{}\fi\par}{}{}
\makeatother
% Allow footnotes in longtable head/foot
\IfFileExists{footnotehyper.sty}{\usepackage{footnotehyper}}{\usepackage{footnote}}
\makesavenoteenv{longtable}
\usepackage{graphicx}
\makeatletter
\def\maxwidth{\ifdim\Gin@nat@width>\linewidth\linewidth\else\Gin@nat@width\fi}
\def\maxheight{\ifdim\Gin@nat@height>\textheight\textheight\else\Gin@nat@height\fi}
\makeatother
% Scale images if necessary, so that they will not overflow the page
% margins by default, and it is still possible to overwrite the defaults
% using explicit options in \includegraphics[width, height, ...]{}
\setkeys{Gin}{width=\maxwidth,height=\maxheight,keepaspectratio}
% Set default figure placement to htbp
\makeatletter
\def\fps@figure{htbp}
\makeatother
\setlength{\emergencystretch}{3em} % prevent overfull lines
\providecommand{\tightlist}{%
  \setlength{\itemsep}{0pt}\setlength{\parskip}{0pt}}
\setcounter{secnumdepth}{5}

\title{A Minimal Book Example}
\author{Nikos Bosse}
\date{2020-07-12}

\begin{document}
\maketitle

{
\setcounter{tocdepth}{1}
\tableofcontents
}
\hypertarget{prerequisites}{%
\chapter{Prerequisites}\label{prerequisites}}

Placeholder

\hypertarget{introduction}{%
\chapter{Introduction}\label{introduction}}

\hypertarget{motivation}{%
\section{Motivation}\label{motivation}}

To obtain accurate knowledge of the future has always been important for policy makers. The demand for predictions of the future trajectory of a disease has been supercharged with the rise of the novel coronavirus SARS-CoV-2. The virus that emerged in Wuhan in January 2020 has by now infected XXXXX and killed over XXXXX people in XXXXX different countries.

In order to get a feeling for the future trajectory of the epidemic, researchers have looked at different quantities like the number of cases, the number of deaths, the expected number of people each infected case is going to infect themselves (\(R\)) and the doubling or halving time of the number of cases.

\hypertarget{outline}{%
\section{Outline}\label{outline}}

This master thesis is going to

\begin{itemize}
\tightlist
\item
  give an overview of the different quantities one can look at
\item
  detail different strategies to measure them / pros and cons to know them
\item
  look at forecasting

  \begin{itemize}
  \tightlist
  \item
    what do we want to measure
  \item
    quantile vs.~point vs.~distributional forecasts
  \end{itemize}
\item
  Look at scoring forecasts
\item
  Look at ensembling

  \begin{itemize}
  \tightlist
  \item
    QRA
  \item
    Stacking
  \end{itemize}
\item
  present the different models we used to forecast US deaths
\item
  present results for the different models and the ensemble
\end{itemize}

\hypertarget{ensembling}{%
\chapter{Ensembling}\label{ensembling}}

The following chapter gives an introduction to the idea of model ensembling and

\hypertarget{theoretical-idea}{%
\section{Theoretical idea}\label{theoretical-idea}}

different models make different independent errors which can be avaraged away.

\hypertarget{different-ensembling-strategies}{%
\section{Different ensembling strategies}\label{different-ensembling-strategies}}

\hypertarget{equally-weighted-ensembe}{%
\subsection{Equally weighted ensembe}\label{equally-weighted-ensembe}}

\hypertarget{mean-average}{%
\subsubsection{mean = average}\label{mean-average}}

\hypertarget{mean-equal-weights-mixture}{%
\subsubsection{mean = equal weights mixture}\label{mean-equal-weights-mixture}}

The simplest way to do ensembling is to take equal weights

\hypertarget{qra-ensembling}{%
\subsection{QRA ensembling}\label{qra-ensembling}}

\hypertarget{crps-ensembling}{%
\subsection{CRPS ensembling}\label{crps-ensembling}}

\hypertarget{know-your-enemy---tracking-a-pandemic}{%
\chapter{Know your enemy - tracking a pandemic}\label{know-your-enemy---tracking-a-pandemic}}

Placeholder

\hypertarget{estimating-cases}{%
\section{Estimating cases}\label{estimating-cases}}

\hypertarget{estimating-deaths}{%
\section{Estimating deaths}\label{estimating-deaths}}

\hypertarget{transferring-between-different-quantities}{%
\section{Transferring between different quantities}\label{transferring-between-different-quantities}}

\hypertarget{estimating-r}{%
\section{\texorpdfstring{Estimating \(R\)}{Estimating R}}\label{estimating-r}}

\hypertarget{estimating-r_t}{%
\subsection{\texorpdfstring{Estimating \(R_t\)}{Estimating R\_t}}\label{estimating-r_t}}

\hypertarget{doubling-and-halving-times}{%
\subsection{Doubling and halving times}\label{doubling-and-halving-times}}

\hypertarget{estimating-r_t-from-cases}{%
\subsection{\texorpdfstring{Estimating \(R_t\) from cases}{Estimating R\_t from cases}}\label{estimating-r_t-from-cases}}

\hypertarget{forecasting-models}{%
\chapter{Forecasting models}\label{forecasting-models}}

chapter gives an overview over different models used to forecast

\hypertarget{forecasting-via-r_t}{%
\section{\texorpdfstring{Forecasting via \(R_t\)}{Forecasting via R\_t}}\label{forecasting-via-r_t}}

\hypertarget{direct-timeseries-forecasts}{%
\section{Direct timeseries forecasts}\label{direct-timeseries-forecasts}}

\hypertarget{arima}{%
\subsection{ARIMA}\label{arima}}

\hypertarget{state-space-models}{%
\subsection{State space models}\label{state-space-models}}

\hypertarget{forecasting-and-evaluation}{%
\chapter{Forecasting and evaluation}\label{forecasting-and-evaluation}}

The following chapter will give a quick overview of different types of forecasts and will detail strategies to score and evaluate these forecasts.

\hypertarget{types-of-forecasts}{%
\section{Types of forecasts}\label{types-of-forecasts}}

Suppose we were interested in the number of Covid-19 cases in December 2020. We can forecast this quantity in a variety of different ways. We could make a point forecasts and predict that the number of cases will be 50,000 cases. We could also quantifiy our uncertainty and state a variance or standard deviation. Note that this, in essence is also a point forecast. In the first instance, we tried to estimate the mean of the underlying data generating distribution, in the second we estimated its variance.

Ideally, we would like to predict the entire unknown data generating distribution to quantify our uncertainty explicitly. This is called a probabilistic forecast. We could for example make a forecast that the number of cases in December 2020 follows a discretised normal distribution with a mean of 50,000 and a variance of 2,000. If we wanted to model uncertainty differently, maybe a different distribution like poisson or negative binomial could be appropriate. Instead of explicitly specifying a predictive distribution, we can make our forecasts in the form of predictive samples. This is especially handy as we can use MCMC algorithms to generate predictions if no analytical solution is available. The downside is that predictive samples take a lot of storage space. They also come with a loss of precision that is especially pronounced in the tails of the predictive distribution, where we need quite a lot of samples to characterise the distribution accurately.

To circumvent these problems, we may decide to store the quantiles of the predictive distribution. Quantile forecasts can easily be obtained from explicit distributional forecasts as well as from predictive samples.

Note that we could also in principle state our forecasts in a binary way. We could for example ask: ``Will the number of cases in December 2020 exceed 50,000?'' and give a probability that this will happen. This type of forecasting is common in many Machine Learning and classification problems, but is beyond the scope of this thesis.

\hypertarget{the-forecasting-paradigm}{%
\section{The forecasting paradigm}\label{the-forecasting-paradigm}}

What is a good forecast? We want our forecasts be as precises and as close to the true value, as possible. But it is not immediately clear how deviations should be taken into account. I.e. be very correct most of the time and very wrong sometimes? Or better to be a bit wrong most of the time? --\textgreater{} depends on the goal of the forecaster. Different metrics focus on different aspects.

The general forecastig paradmigm as formulated by Gneiting: We want to increase sharpness subject to calibration.

\hypertarget{scoring-forecasts}{%
\section{Scoring Forecasts}\label{scoring-forecasts}}

Having made a prediction, we need to evaluate it to see if it is any good. Quite a few metrics and scoring rules are available to help with this task.

\hypertarget{scoring-point-forecasts}{%
\subsection{Scoring point forecasts}\label{scoring-point-forecasts}}

Numerous different metrics are available to help evaluate the quality of point forecasts. The package \texttt{metrics} lists XXX. Most important: Mean Squared Error (MSE), Mean Absolute Error (MAE), Mean Absolute Percentage Error (MAPE).

--\textgreater{} bottom line:

\hypertarget{assessing-probabilistic-forecasts}{%
\subsection{Assessing probabilistic forecasts}\label{assessing-probabilistic-forecasts}}

Scoring probabilistic forecasts is more difficult, as the entire predictive distribution has to be taken into account. This can be achieved using so called proper scoring rules CITATION.

Other metrics

\hypertarget{scoring-quantile-forecasts}{%
\subsection{Scoring quantile forecasts}\label{scoring-quantile-forecasts}}

Interval Score

\hypertarget{interval-score}{%
\section{Interval Score}\label{interval-score}}

The Interval Score is a Proper Scoring Rule to score quantile predictions,
following Gneiting and Raftery (2007). Smaller values are better.

The score is computed as

\[ \text{score} = (\text{upper} - \text{lower}) + \\
\frac{2}{\alpha} \cdot (\text{lower} - \text{true_value}) \cdot 1(\text{true_values} < \text{lower}) + \\
\frac{2}{\alpha} \cdot (\text{true_value} - \text{upper}) \cdot
1(\text{true_value} > \text{upper})\]

where \(1()\) is the indicator function and \(\alpha\) is the decimal value that indicates how much is outside the prediction interval. No specific distribution is assumed. One can weigh the score by \(\frac{\alpha}{2}\) such that
the Interval Score converges to the CRPS for increasing number of quantiles. CITATION.

\hypertarget{the-scoringutils-package}{%
\section{The scoringutils package}\label{the-scoringutils-package}}

In order to score the predictions made for the purpose of this thesis, a collection of metrics and proper scoring rules were bundled in an \texttt{R} package called \texttt{scoringutils}. CITATION. In the package included are the above described metrics as well as functionality to automatically score a set of predictions with the appropriate metrics.

\end{document}
